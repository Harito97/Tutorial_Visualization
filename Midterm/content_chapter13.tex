\textbf{Cách thiết kế trực quan hiệu quả}

Mục tiêu của chương này là cung cấp một số \textbf{cách để thiết kế hình
ảnh trực quan thành công} vì nó là hình ảnh truyền tải thông tin mong
muốn đến đối tượng mục tiêu một cách hiệu quả, chính xác. Sẽ có vô số
phương pháp khả thi để ánh xạ các thành phần dữ liệu thành các ảnh.
Tương tự cũng có rất nhiều công cụ tương tác có thể được cung cấp cho
người xem. Việc lựa chọn các kết hợp kỹ thuật hiệu quả nhất không phải
là một quá trình đơn giản.

Một hình ảnh trực quan có thể không hiệu quả vì một số lý do:

\begin{itemize}
\tightlist
\item
  \textbf{quá khó hiểu hoặc phức tạp} để đối tượng người nghe mục tiêu
  có thể hiểu được
\item
  một số \textbf{dữ liệu bị bóp méo, che khuất}
\item
  \textbf{thiếu hỗ trợ} việc \textbf{sửa đổi chế độ xem} hoặc
  \textbf{kiểm soát bản đồ màu}
\item
  một bản trình bày \textbf{không hấp dẫn về mặt thị giác}
\end{itemize}

Chương này trước tiên trình bày các cân nhắc về {[}1{]} \textbf{cách
thiết kế cho các thành phần mà tác giả cảm thấy cần thiết cho một hình
ảnh trực quan} tốt. Sau đó, khám phá {[}2{]} \textbf{vấn đề thường gặp
trong hình ảnh trực quan} và \textbf{một số kỹ thuật để tránh} những vấn
đề này.

\section{Các bước thiết kế trực
quan}\label{cuxe1c-bux1b0ux1edbc-thiux1ebft-kux1ebf-trux1ef1c-quan}

Việc tạo hình ảnh hóa bao gồm quyết định cách \textbf{ánh xạ các trường
dữ liệu thành các hình ảnh}, lựa chọn và triển khai các phương pháp để
\textbf{sửa đổi chế độ xem} và \textbf{chọn lượng dữ liệu cần thiết cho
trực quan hóa}.

\textbf{Thông tin bổ sung liên quan đến dữ liệu} được hiển thị (ví dụ:
nhãn) và \textbf{ánh xạ} (ví dụ: khóa màu) cũng rất \textbf{cần thiết
cho việc diễn giải} và \textbf{phải được tích hợp vào hình ảnh}.

Cuối cùng, ít hữu hình hơn, cần cân nhắc đến \textbf{tính thẩm mỹ tổng
thể} của màn hình \textbf{hiển thị kết quả}. Các phần nhỏ được trình bày
sau đây sẽ làm rõ hơn các bước trong quy trình đã đề cập.

\subsection{Bước 1. Ánh xạ dữ liệu sang biểu diễn trực
quan}\label{bux1b0ux1edbc-1.-uxe1nh-xux1ea1-dux1eef-liux1ec7u-sang-biux1ec3u-diux1ec5n-trux1ef1c-quan}

Để tạo ra hình ảnh trực quan hiệu quả, cần hiểu rõ \textbf{ngữ nghĩa của
dữ liệu và bối cảnh của người xem}. Việc \textbf{chọn cách hiển thị phù
hợp với tư duy của người xem} sẽ giúp họ \textbf{dễ dàng hiểu hình ảnh
hơn}. Ngoài ra, nhà thiết kế \textbf{nên nhất quán để tránh sự nhầm
lẫn}. Ánh xạ dữ liệu trực quan tốt sẽ giúp người xem diễn giải nhanh hơn
vì không phải mất thời gian để hiểu.

Ví dụ, trong \href{...}{hình 1}, hình ảnh các hành tinh được sử dụng để
vẽ mối quan hệ giữa khoảng cách từ hành tinh đến Mặt Trời và thời gian
quỹ đạo của nó. Fig 1. ()

Việc ánh xạ các \textbf{thuộc tính dữ liệu không gian}, để diễn giải bề
mặt cầu 3 chiều của Trái Đất thành 1 mặt phẳng 2D là cách ánh xạ phổ
biến và trực quan nhất được tìm thấy trong các hình ảnh trực quan. Ví dụ
này là sớm nhất để tận dụng khả năng của con người trong việc liên hệ vị
trí trên mặt phẳng 2D với vị trí trong thế giới ba chiều.

Tương tự như vậy, với sự ra đời của hoạt hình, rõ ràng là việc hiển thị
các \textbf{tập dữ liệu chuỗi thời gian} thông qua hoạt hình là khá trực
quan, với lợi thế bổ sung là cho phép thời gian thay đổi cả về tốc độ và
hướng. Fig 2. ()

Một số \textbf{ánh xạ dữ liệu trở nên trực quan hơn khi kết hợp với ngữ
cảnh cụ thể}:

\begin{itemize}
\item
  Ví dụ, \textbf{ánh xạ nhiệt độ sang màu sắc} rất phổ biến, ví dụ xanh
  đỏ đại diện cho nhiệt độ thấp và nhiệt độ cao. Fig 3. ()
\item
  Trong các lĩnh vực như bản đồ học và địa chất, \textbf{màu sắc thường
  được dùng để phân loại đất đai hoặc lớp địa chất}, vì vậy việc chọn
  màu phải phù hợp với bối cảnh ứng dụng. Fig 4. ()
\item
  \textbf{Chiều cao hoặc độ dài} cũng có thể dùng để \textbf{biểu diễn
  nhiệt độ}, giống như cách chúng ta đọc nhiệt kế. Đối với các bác sĩ,
  việc dùng độ dài để hiển thị áp lực hoặc các giá trị liên quan có thể
  rất tự nhiên. Fig 5. ()
\end{itemize}

Khi \textbf{chọn ánh xạ}, cần \textbf{xem xét tính tương thích giữa
thang dữ liệu và thuộc tính}. Dữ liệu có thứ tự (như tuổi) không nên ánh
xạ sang thuộc tính không có thứ tự (như hình dạng). Tương tự, dữ liệu
không có thứ tự (như quốc gia) không nên ánh xạ sang thuộc tính có thứ
tự (như độ dài). Fig 6. () \& Fig 7. ()

Tuy nhiên, \textbf{đôi khi cũng thú vị khi kiểm tra dữ liệu bằng các ánh
xạ không trực quan}, vì hình ảnh kết quả có thể tiết lộ một thuộc tính
thú vị trong dữ liệu. Ví dụ, ánh xạ thời gian để tô màu dọc theo một
đường vạch có thể tiết lộ các biến thể về tốc độ hạt mà nếu không thì có
thể khó phát hiện. Do đó, \textbf{một nguyên tắc chung} hữu ích là
\textbf{thiết lập các ánh xạ mặc định dựa trên lựa chọn trực quan nhất
theo người xem thông thường}, nhưng, đặc biệt đối với các tác vụ khám
phá, cho phép người xem tùy chỉnh nhiều ánh xạ trực quan khác nhau.

\subsection{Bước 2. Chọn và sửa đổi chế độ xem cho phù
hợp}\label{bux1b0ux1edbc-2.-chux1ecdn-vuxe0-sux1eeda-ux111ux1ed5i-chux1ebf-ux111ux1ed9-xem-cho-phuxf9-hux1ee3p}

Ngoại trừ các tập dữ liệu khá đơn giản, \textbf{một chế độ xem hiếm khi
đủ để truyền tải tất cả thông tin} chứa trong dữ liệu. Như vậy điều quan
trọng là \textbf{phải có thể dự đoán các loại chế độ xem} mà được sử
dụng nhiều nhất bởi người xem thông thường và sau đó \textbf{cung cấp
trực quan cách điều khiển cài đặt, tùy chỉnh các dạng xem mà người xem
cảm thấy phù hợp}.

Cần ghi nhớ \textbf{chế độ xem phù hợp phụ thuộc vào}:

\begin{itemize}
\tightlist
\item
  loại dữ liệu được trình bày
\item
  và nhiệm vụ gắn liền với sự trực quan hóa.
\item
  Mỗi chế độ xem phải có nhãn, cung cấp thông tin rõ ràng, đầy đủ
\item
  và người xem cần ít hành động nhất có thể để sửa sang được chế độ xem
  mà họ thấy phù hợp.
\end{itemize}

Sau đây là ví dụ \textbf{một số hành động} của người xem để \textbf{sửa
đổi chế độ xem}:

\begin{itemize}
\tightlist
\item
  \textbf{Cuộn và phóng to}: Cần thiết \textbf{khi không thể hiển thị
  toàn bộ dữ liệu} ở độ phân giải mong muốn.
\item
  \textbf{Điều khiển bảng màu}: Luôn cần thiết, ít nhất nên \textbf{hỗ
  trợ một số bảng màu khác nhau} để cho phép người xem điều chỉnh màu
  sắc riêng lẻ hoặc toàn bộ bảng màu.
\item
  \textbf{Điều khiển ánh xạ}: Cho phép người xem \textbf{chuyển đổi giữa
  các cách hiển thị khác nhau của cùng một dữ liệu}. Lý do vì một số đặc
  điểm có thể bị ẩn trong ánh xạ này nhưng lại nổi bật trong ánh xạ
  khác.
\item
  \textbf{Điều khiển tỷ lệ}: Cho phép người xem \textbf{thay đổi phạm vi
  và phân phối giá trị cho một trường dữ liệu trước khi ánh xạ}. Việc
  lọc và cắt dữ liệu cũng \textbf{giúp người xem tập trung vào các tập
  con cụ thể}.
\item
  \textbf{Điều khiển mức độ chi tiết}: Cung cấp \textbf{khả năng loại bỏ
  hoặc làm nổi bật chi tiết}, hỗ trợ các góc nhìn ở các mức độ khác
  nhau.
\end{itemize}

\textbf{Lưu ý} cho việc thiết kế chế độ xem và tùy chọn sửa đổi cho
người xem:

\begin{itemize}
\tightlist
\item
  Trong mọi trường hợp, điều cần thiết là các \textbf{thao tác xem phải
  đơn giản, dễ nhớ} cho người xem và \textbf{cung cấp những thông tin
  phù hợp, chính xác} cho nhiệm vụ.
\item
  Nếu có thể, \textbf{thao tác sửa đổi chế độ xem trực tiếp} (trực tiếp
  có thể thay đổi chế độ xem bằng các thao tác đơn giản: click, gõ phím,
  \ldots) \textbf{thường được ưa thích}.
\end{itemize}

\subsection{Bước 3. Xác định mật độ thông tin phù
hợp}\label{bux1b0ux1edbc-3.-xuxe1c-ux111ux1ecbnh-mux1eadt-ux111ux1ed9-thuxf4ng-tin-phuxf9-hux1ee3p}

Khi thiết kế trực quan hóa, quan trọng là \textbf{xác định lượng thông
tin cần hiển thị}. Có hai hệ quả:

\begin{itemize}
\item
  \textbf{Đồ họa thừa}: \textbf{Có quá ít thông tin để trình bày}. Ví
  dụ, việc chỉ cần hiển thị tỷ lệ nam và nữ có thể chỉ dùng một con số.
  Một số đồ họa có thể cố gắng ``làm đầy'' thêm thông tin bằng cách hiển
  thị nhiều giá trị hơn, nhưng trong \textbf{những trường hợp này, chỉ
  cần hiển thị các giá trị dưới dạng văn bản sẽ hiệu quả hơn}.
\item
  \textbf{Thông tin thừa}: \textbf{Việc trực quan hóa chứa quá nhiều
  thông tin} \(\to\) nó có thể \textbf{gây nhầm lẫn và khó hiểu}. Thông
  tin quan trọng có thể bị mất trong một giao diện lộn xộn, khiến người
  xem \textbf{khó xác định nơi cần chú ý}.
\end{itemize}

\textbf{Giải pháp} cho \textbf{vấn đề quá tải thông tin}:

\begin{itemize}
\tightlist
\item
  \textbf{Tùy chọn hiển thị}: Cho phép người xem \textbf{bật hoặc tắt
  các thành phần khác nhau}, giúp họ \textbf{tập trung vào thông tin
  quan trọng}.
\item
  \textbf{Nhiều màn hình}: Sử dụng các \textbf{màn hình riêng biệt} để
  \textbf{hiển thị 1 nhóm thông tin cụ thể} mà không gây lộn xộn.
\item
  \textbf{Lọc dữ liệu}: \textbf{Loại bỏ} các điểm \textbf{dữ liệu không
  quan trọng} để người xem chỉ \textbf{tập trung vào các phần có ý
  nghĩa}.
\item
  \textbf{Tỉ lệ}: Điều chỉnh \textbf{kích thước của một số dữ liệu cho
  phân bổ trên không gian màn hình hợp lý} hơn.
\end{itemize}

Những giải pháp này giúp tối ưu hóa trực quan hóa, đảm bảo rằng người
xem dễ dàng hiểu và tương tác với thông tin.

\subsection{Bước 4. Thêm vào các từ khóa, nhãn và chú
thích}\label{bux1b0ux1edbc-4.-thuxeam-vuxe0o-cuxe1c-tux1eeb-khuxf3a-nhuxe3n-vuxe0-chuxfa-thuxedch}

Một vấn đề phổ biến trong trực quan hóa là \textbf{thiếu thông tin hỗ
trợ để người xem có thể hiểu rõ và chính xác} thông tin truyền tải. Để
giải quyết vấn đề đó, chúng ta \textbf{cần cung cấp các yếu tố sau}:

\begin{itemize}
\tightlist
\item
  \textbf{Chú thích} chi tiết: \textbf{Giải thích dữ liệu} và
  \textbf{phép ánh xạ dữ liệu sang biểu đồ trực quan} được sử dụng.
\item
  \textbf{Lưới và dấu tick}: Hiển thị \textbf{các giá trị và phạm vi}
  của các trường số \textbf{khi cần đánh giá chính xác}.
\item
  \textbf{Nhãn trục}: Ghi rõ \textbf{đơn vị đo lường trên các trục}.
\item
  \textbf{Chú giải ký hiệu}: Cung cấp \textbf{từ khóa cho các ký hiệu},
  nằm ở viền hoặc trong một widget riêng.
\item
  \textbf{Giải thích màu sắc}: \textbf{Ý nghĩa của màu sắc} sử dụng
  \textbf{trong phép trực quan}, như thanh màu có nhãn.
\end{itemize}

\subsection{Bước 5. Điều chỉnh màu sắc được sử dụng cho phù
hợp}\label{bux1b0ux1edbc-5.-ux111iux1ec1u-chux1ec9nh-muxe0u-sux1eafc-ux111ux1b0ux1ee3c-sux1eed-dux1ee5ng-cho-phuxf9-hux1ee3p}

\textbf{Màu sắc thường bị lạm dụng} trong các biểu đồ, \textbf{dẫn đến
sự nhầm lẫn hoặc diễn giải sai}.

Thêm nữa, việc \textbf{chọn sai bảng màu} hoặc cố gắng \textbf{truyền
tải quá nhiều thông tin qua màu sắc} có thể làm \textbf{giảm hiệu quả
trực quan hóa}.

Đặc biệt, do \textbf{màu sắc có thể bị ảnh hưởng bởi môi trường quan
sát} và \textbf{nhiều người mắc chứng mù màu}, điều này càng làm phức
tạp quá trình thiết kế.

Dưới đây là \textbf{hướng dẫn sử dụng màu sắc hiệu quả} trong biểu đồ:

\begin{itemize}
\item
  \textbf{Giới hạn số lượng màu sắc}, tránh làm rối mắt người xem.
\item
  Sử dụng \textbf{thêm phương pháp ánh xạ bổ sung}, ví dụ như kết hợp cả
  màu sắc và kích thước, để dễ dàng truyền đạt thông tin.
\item
  Ngoài ra, khi tạo bảng màu cho dữ liệu số, có thể \textbf{thay đổi sắc
  độ} (hue) và \textbf{độ sáng} (lightness) để giúp dễ phân biệt các mục
  (entry) với ít máu sắc nhất.
\end{itemize}

\subsection{Bước 6. Bước cuối cùng, đảm bảo thẩm
mỹ}\label{bux1b0ux1edbc-6.-bux1b0ux1edbc-cuux1ed1i-cuxf9ng-ux111ux1ea3m-bux1ea3o-thux1ea9m-mux1ef9}

Đây là bước chúng ta thực hiện \textbf{cân bằng giữa chức năng và hình
thức}. Một biểu đồ tốt nên \textbf{vừa cung cấp thông tin vừa hấp dẫn về
mặt hình ảnh}.

Dưới đây là \textbf{hướng dẫn nâng cao tính thẩm mỹ}:

\begin{itemize}
\tightlist
\item
  \textbf{Tập trung}: Nên làm \textbf{nổi bật những phần quan trọng
  nhất} của biểu đồ để người xem \textbf{biết nên chú ý vào đâu}. Nếu
  không có sự nhấn mạnh thích hợp, thông tin quan trọng có thể bị bỏ
  qua.
\item
  \textbf{Cân bằng}: Sử dụng \textbf{không gian màn hình hợp lý}. Đặt
  các \textbf{thành phần chính ở trung tâm} và \textbf{tránh nổi bật
  đường viền} hoặc \textbf{khu vực ít quan trọng}.
\item
  \textbf{Đơn giản}: Giữ \textbf{lượng thông tin thể hiện trực quan ở
  mức vừa đủ bằng những thiết kế tối giản}. Nên dùng những hình ảnh đơn
  giản và tránh dùng các thiết kế phức tạp nếu những thiết kế đơn giản
  hơn có thể truyền đạt thông điệp tương tự. Một kỹ thuật hữu ích là
  loại bỏ từng yếu tố và xem xét liệu việc mất thông tin có chấp nhận
  được không. \textbf{Ghi nhớ ``tối giản'' - đơn giản mà vẫn đạt yêu cầu
  đặt ra}.
\end{itemize}

\section{Các lỗi sai hay gặp khi thiết kế trực
quan}\label{cuxe1c-lux1ed7i-sai-hay-gux1eb7p-khi-thiux1ebft-kux1ebf-trux1ef1c-quan}

Ngay cả khi \textbf{tuân thủ quy trình các bước} thiết kế trực quan khoa
học nêu trên, \textbf{vẫn có thể có vài lỗi sai}, vấn đề gặp phải.

Những vấn đề này do nhiều lý do:

\begin{itemize}
\tightlist
\item
  liên quan đến quyết định \textbf{nên trực quan hóa điều gì}
\item
  và \textbf{phương pháp ánh xạ trực quan nào là phù hợp} nhất để sử
  dụng.
\item
  Một số vấn đề liên quan đến việc \textbf{bóp méo dữ liệu}, dù cố ý hay
  vô tình, có thể dẫn đến sự hiểu sai.
\item
  Các vấn đề khác bao gồm việc \textbf{che giấu dữ liệu thực sự} đằng
  sau các phiên bản ``đã được làm sạch'' hoặc các đồ họa phụ trợ quá
  thừa thãi đi kèm.
\end{itemize}

Trong tất cả các trường hợp này, vẫn có thể thực hiện các bước để cải
thiện chất lượng và tính ``trung thực'' của hình ảnh trực quan. Sau đây
chúng ta sẽ bắt đầu.

\subsection{Hình ảnh gây hiểu
lầm}\label{huxecnh-ux1ea3nh-guxe2y-hiux1ec3u-lux1ea7m}

Một trong những \textbf{quy tắc quan trọng} nhất của trực quan hóa là nó
phải \textbf{mô tả chính xác dữ liệu}. Tuy nhiên, trong suốt lịch sử, có
nhiều ví dụ về việc trực quan hóa dữ liệu bị bóp méo được sử dụng để
thay đổi quan điểm và lừa dối khán giả. Những ``ảnh giả'' này có thể
xuất hiện ở khắp mọi nơi, từ các tạp chí danh tiếng cho đến các danh mục
của công ty.

Trong phần này, chúng ta sẽ xác định \textbf{một số chiến lược phổ biến}
trong việc \textbf{tạo ra các hình ảnh trực quan gây hiểu lầm},
\textbf{không phải để} người đọc \textbf{áp dụng}, \textbf{mà là để
tránh}!

\subsection{Làm sạch dữ liệu}\label{luxe0m-sux1ea1ch-dux1eef-liux1ec7u}

Dữ liệu thô thường có nhiều yếu tố ``lộn xộn'', chẳng hạn như những điểm
dữ liệu bất thường (outliers) hoặc lỗi thu thập dữ liệu. \textbf{Trong
quá trình làm trực quan}, người làm trực quan hóa có thể \textbf{bị cám
dỗ loại bỏ những điểm dữ liệu không hợp lý} để giúp hình ảnh trông dễ
nhìn hơn hoặc để loại bỏ các dữ liệu không phù hợp với quan điểm mà họ
muốn thể hiện.

Ví dụ:

Hình (a) Dữ liệu thô cho thấy thiếu mối tương quan (b) Dữ liệu đã làm
sạch tiết lộ mối tương quan giả

Loại bỏ các điểm ngoại lệ là một cách phổ biến trong trường hợp này. Các
điểm ngoại lệ là những giá trị không bình thường trong dữ liệu, có thể
là do lỗi kỹ thuật hoặc là thực tế hiếm hoi nhưng hợp lệ.

\textbf{Khi những điểm ngoại lệ} này \textbf{bị loại bỏ không có lý do
hợp lý} hoặc \textbf{không thông báo cho người xem}, hình ảnh trực quan
có thể trở nên \textbf{thiên lệch và gây hiểu lầm}.

\textbf{Giải pháp}: \textbf{Trừ khi có bằng chứng} rằng các điểm ngoại
lệ là kết quả của lỗi kỹ thuật, \textbf{chúng không nên bị loại bỏ một
cách tùy tiện}. Đồng thời, người thiết kế \textbf{cần cung cấp} cho
người xem tùy chọn \textbf{hiển thị hoặc ẩn} những \textbf{điểm ngoại
lệ} này để đảm bảo tính minh bạch của dữ liệu.

\subsection{Tỷ lệ không cân
đối}\label{tux1ef7-lux1ec7-khuxf4ng-cuxe2n-ux111ux1ed1i}

Tỷ lệ là một công cụ mạnh mẽ trong hình ảnh trực quan, vì việc
\textbf{lựa chọn kỹ lưỡng các yếu tố tỷ lệ có thể giúp làm lộ ra các mẫu
và cấu trúc không thể nhìn thấy ở các góc nhìn không được tỷ lệ hóa}.

Tuy nhiên, \textbf{tỷ lệ cũng có thể bị lợi dụng để lừa dối người xem}.
Ví dụ, thay đổi tỷ lệ của các trục hay kích thước các yếu tố đồ họa có
thể làm cho một xu hướng dường như mạnh hơn hoặc yếu hơn so với thực tế.
Điều này tạo ra một sự sai lệch, khiến người xem đánh giá không chính
xác sự thay đổi trong dữ liệu.

Edward Tufte, một chuyên gia về hình ảnh trực quan, đưa ra khái niệm
``\textbf{lie factor}'', chỉ tỷ lệ giữa sự thay đổi thực tế trong dữ
liệu và sự thay đổi trong hình ảnh. Nếu \textbf{tỷ lệ này quá cao hoặc
thấp, hình ảnh đó có thể bị coi là lừa dối}.

\textbf{Giải pháp}: Cung cấp \textbf{các công cụ như đường kẻ},
\textbf{điểm tham chiếu}, hoặc \textbf{chỉ báo điểm gốc} để người xem có
thể hiểu rõ hơn về \textbf{cách dữ liệu được tỷ lệ hóa} và có thể
\textbf{so sánh chính xác hơn}.

\subsection{Biến dạng phạm vi}\label{biux1ebfn-dux1ea1ng-phux1ea1m-vi}

\textbf{Người xem thường có kỳ vọng về phạm vi của một chiều dữ liệu
nhất định}.

Do con người có khả năng phán đoán tương đối rất mạnh mẽ, tức là chúng
ta thường \textbf{so sánh các đối tượng với nhau} thay vì chỉ đánh giá
\textbf{từng đối tượng một cách độc lập}, việc \textbf{thay đổi điểm
gốc} -- tức là \textbf{giá trị mà từ đó chúng ta bắt đầu so sánh} -- có
thể \textbf{ảnh hưởng cách chúng ta hiểu} và diễn giải thông tin.

\textbf{Giải pháp}: Người thiết kế \textbf{cần minh bạch với người xem
về điểm gốc của biểu đồ} và \textbf{lý do thay đổi nếu có}. Việc cho
phép người xem \textbf{tùy chọn điều chỉnh điểm gốc cũng là một cách}
giúp họ \textbf{không bị lừa dối bởi phạm vi dữ liệu}.

\subsection{\texorpdfstring{\textbf{Lạm dụng tính đa
chiều}}{Lạm dụng tính đa chiều}}\label{lux1ea1m-dux1ee5ng-tuxednh-ux111a-chiux1ec1u}

\textbf{Hình ảnh trực quan càng nhiều chiều thì càng dễ gây hiểu lầm}.
Điều này là do con người gặp nhiều khó khăn hơn khi đánh giá thể tích (3
chiều) so với diện tích (2 chiều), và diện tích lại khó hơn chiều dài (1
chiều).

Ví dụ: Nếu một giá trị vô hướng (scalar) được đánh giá thông qua đồ thị
3D, người xem có thể gặp khó khăn trong việc đánh giá chính xác sự khác
biệt giữa các yếu tố, dẫn đến khả năng hiểu sai về mối quan hệ dữ liệu.

\textbf{Giải pháp}: Việc \textbf{đơn giản hóa hình ảnh trực quan} sẽ
giúp người xem dễ dàng nắm bắt thông tin hơn. Chỉ nên sử dụng nhiều
chiều dữ liệu khi thật sự cần thiết và đảm bảo rằng người xem có thể
hiểu rõ ý nghĩa của chúng.

\subsection{Hình ảnh vô nghĩa}\label{huxecnh-ux1ea3nh-vuxf4-nghux129a}

Hình ảnh trực quan được thiết kế để truyền tải thông tin, và điều quan
trọng là \textbf{thông tin đó phải có ý nghĩa}.

Dưới đây là các cách thường được dùng để ``đánh lừa'' tư duy của chúng
ta bằng cách gây hiểu nhầm tương quan các chiều (biến) của dữ liệu.

\subsection{Kết hợp các quan hệ ngẫu nhiên như là một quan hệ nhân quả
có tương
quan}\label{kux1ebft-hux1ee3p-cuxe1c-quan-hux1ec7-ngux1eabu-nhiuxean-nhux1b0-luxe0-mux1ed9t-quan-hux1ec7-nhuxe2n-quux1ea3-cuxf3-tux1b0ux1a1ng-quan}

Các hình ảnh trực quan thường được tạo ra bằng cách kết hợp các tập dữ
liệu từ nhiều nguồn khác nhau. Tuy nhiên, \textbf{việc kết hợp các thành
phần không liên quan vào một hình ảnh duy nhất rất dễ xảy ra}, và từ đó
có thể nhận thấy \textbf{những gì dường như là một cấu trúc, có liên
quan nhưng thực tế sai bét, vô nghĩa}.

Trong trường hợp này, \textbf{mối quan hệ ngẫu nhiên bị nhầm lẫn với mối
quan hệ nhân quả}. Khi quyết định kết hợp dữ liệu nào, điều quan trọng
là phải \textbf{đảm bảo rằng có một logic nhất định trong sự kết hợp
đó}.

Một trong những vấn đề được tìm thấy trong các quy trình nhận diện mẫu
phân tích dữ liệu là \textbf{những mối quan hệ không liên quan thường
được phát hiện và báo cáo}, \textbf{sau đó phải được loại bỏ bởi các
chuyên gia trong lĩnh vực}.

\subsection{So sánh trong phạm vi không gian và thời gian không tương
quan}\label{so-suxe1nh-trong-phux1ea1m-vi-khuxf4ng-gian-vuxe0-thux1eddi-gian-khuxf4ng-tux1b0ux1a1ng-quan}

\textbf{Một yếu tố khác} cần được xem xét là \textbf{sự tương thích giữa
phạm vi thời gian và không gian} của dữ liệu khi so sánh. Ví dụ, không
nên so sánh doanh số bán hàng của một sản phẩm cụ thể trong một năm ở
một khu vực nhất định của đất nước với doanh số bán hàng của cùng sản
phẩm đó ở một khu vực và năm khác, trừ khi có giả thuyết về sự thay đổi
trong xu hướng quan tâm đến sản phẩm đó.

\subsection{So sánh trong phạm vi đơn vị không chuẩn
hóa}\label{so-suxe1nh-trong-phux1ea1m-vi-ux111ux1a1n-vux1ecb-khuxf4ng-chuux1ea9n-huxf3a}

\textbf{Sự tương thích về đơn vị} cũng cần được kiểm tra khi tạo tập dữ
liệu cho hình ảnh trực quan. Ví dụ, các sản phẩm thực phẩm được đo bằng
giá theo thể tích thường bị trộn lẫn với những sản phẩm được đo bằng giá
theo trọng lượng. Một \textbf{hình ảnh trực quan hiệu quả có thể chuẩn
hóa cả hai đơn vị khác nhau về chung 1 thang đo} về giá theo khẩu phần
ăn.

\subsection{Gán thứ tự cho các đối tượng dữ liệu không có quan hệ thứ
tự}\label{guxe1n-thux1ee9-tux1ef1-cho-cuxe1c-ux111ux1ed1i-tux1b0ux1ee3ng-dux1eef-liux1ec7u-khuxf4ng-cuxf3-quan-hux1ec7-thux1ee9-tux1ef1}

\textbf{Dữ liệu phân loại} là loại dữ liệu \textbf{không có thứ tự tự
nhiên}, chẳng hạn như tên công ty hoặc loại sản phẩm. Khi các dữ liệu
này được \textbf{biểu diễn trên một đồ thị với vị trí cụ thể}, có thể
tạo ra cảm giác rằng \textbf{chúng có thứ tự hoặc có thể so sánh như dữ
liệu liên tục}. Ví dụ: Giả sử bạn có một danh sách các công ty và bạn
muốn biểu diễn chúng trên một đồ thị. Nếu bạn gán mỗi công ty một vị trí
trên trục x của đồ thị và cố gắng vẽ một đường thẳng hoặc đường cong qua
các điểm này, bạn đang áp dụng một phép toán liên tục (vẽ đường) cho dữ
liệu phân loại (tên công ty). Điểm mấu chốt là \textbf{cần có sự cân
nhắc về mặt ngữ nghĩa của hình ảnh trực quan để đảm bảo rằng nó có ý
nghĩa logic}.

\subsection{Mất dữ liệu}\label{mux1ea5t-dux1eef-liux1ec7u}

``\textbf{Chart junk}'' là thuật ngữ được Tufte sử dụng để chỉ các
\textbf{yếu tố đồ họa bổ sung không cần thiết}, làm cho hình ảnh trở nên
phức tạp hơn mà \textbf{không hỗ trợ việc diễn giải dữ liệu}. Những yếu
tố này có thể gây rối mắt, làm người xem khó tập trung vào dữ liệu
chính.

Việc xác định lượng đồ họa bổ sung cần đưa vào một hình ảnh trực quan có
thể là một quá trình khó khăn, vì nhà thiết kế có thể \textbf{không hiểu
rõ nhu cầu của tất cả người xem} tiềm năng.

Khác với các biểu đồ tĩnh của Tufte, \textbf{các hình ảnh trực quan hiện
đại có thể linh hoạt và tùy biến, cho phép người xem điều chỉnh loại và
mật độ thông tin hỗ trợ}. Trong một số nhiệm vụ, người xem có thể chuyển
đổi giữa cái nhìn tổng quát định tính và phân tích định lượng.

Đối với các \textbf{cái nhìn định tính}, việc \textbf{làm rõ dữ liệu} là
rất quan trọng, trong khi đối với \textbf{phân tích định lượng}, các
\textbf{công cụ giúp định lượng} các yếu tố là rất cần thiết.

Một quy tắc tốt là cung cấp đủ \textbf{công cụ để đáp ứng nhu cầu định
lượng} của người xem, nhưng đồng thời cũng cho phép họ có \textbf{tùy
chọn tắt hoặc bật} này để giảm bớt sự lộn xộn trong hình ảnh trực quan.

\subsection{Dữ liệu thô so với dữ liệu suy
diễn}\label{dux1eef-liux1ec7u-thuxf4-so-vux1edbi-dux1eef-liux1ec7u-suy-diux1ec5n}

Các cách phổ biến là thay thế hoàn toàn dữ liệu thô sẵn có bằng cách:

\subsection{Áp đặt một mô hình khớp suy
diễn}\label{uxe1p-ux111ux1eb7t-mux1ed9t-muxf4-huxecnh-khux1edbp-suy-diux1ec5n}

Một hình thức phổ biến trong hình ảnh trực quan là tính toán \textbf{một
mô hình phân tích của dữ liệu} bằng cách sử dụng \textbf{phép khớp đường
cong hoặc bề mặt} để đạt được \textbf{kết quả có tính thẩm mỹ} hơn. Tuy
nhiên, đây là \textbf{một hình thức bóp méo sự thật} và có thể dẫn đến
\textbf{những giả định và kết luận sai lầm} từ phía người quan sát.
Trong một số hình ảnh trực quan, thường loại bỏ tất cả dữ liệu thô và
chỉ hiển thị kết quả xấp xỉ mượt mà từ dữ liệu đó. Điều này buộc người
xem phải tin rằng phép xấp xỉ là sự phản ánh chính xác của dữ liệu,
nhưng điều này không phải lúc nào cũng đúng khi nhà thiết kế áp dụng các
thuật toán khớp số liệu một cách mù quáng. Tốt nhất là \textbf{nên hiển
thị cả dữ liệu thô và mô hình đã khớp trước}, đồng thời \textbf{cho phép
người xem} giảm mức độ hiển thị hoặc \textbf{lọc bỏ một trong hai tùy
theo yêu cầu}.

\subsection{Áp đặt 1 tập được lấy lại
mẫu}\label{uxe1p-ux111ux1eb7t-1-tux1eadp-ux111ux1b0ux1ee3c-lux1ea5y-lux1ea1i-mux1eabu}

Một hình thức khác của việc làm sạch dữ liệu là quá trình \textbf{lấy
mẫu lại (resampling)}, trong đó dữ liệu thô, được định vị trên lưới thưa
thớt hoặc ngẫu nhiên, được sử dụng để tạo ra dữ liệu dày hơn hoặc trên
lưới có khoảng cách đều đặn. Điều này có thể \textbf{mang lại một hình
ảnh trực quan phong phú hơn}, \textbf{gần với mẫu liên tục}, nhưng nó
cũng đánh lừa người xem \textbf{khiến họ tin rằng tập dữ liệu lớn hơn
nhiều so với thực tế}. \textbf{Mật độ lấy mẫu lại càng cao}, người xem
\textbf{càng dễ hiểu sai dữ liệu}, trừ khi hiện tượng đang quan sát có
ít biến động.

Hình 13.21 Sự khác biệt trong cách lấy mẫu và nội suy từ cùng một tập dữ
liệu có thể tạo ra những hình ảnh rất khác nhau. Trong trường hợp này,
một số chi tiết trong hình ảnh bên phải không được nhìn thấy trong hình
ảnh bên trái.

\textbf{Lấy mẫu không đầy đủ} cũng là một vấn đề khác. Hậu quả của việc
này là nhiều thông tin, đặc điểm được thể hiện trên dữ liệu thô bị lược
bỏ.

\textbf{Điều quan trọng} là người xem luôn có \textbf{quyền truy cập vào
dữ liệu thô} và \textbf{được thông báo về bất kỳ hoạt động làm sạch},
\textbf{làm mịn} hoặc \textbf{lấy mẫu lại} nào đã được áp dụng.

Trong một số lĩnh vực, chẳng hạn như chẩn đoán hình ảnh (radiology), các
nhà phân tích kiên quyết \textbf{phản đối bất kỳ dạng làm mịn hoặc lọc
dữ liệu nào}, vì có \textbf{nguy cơ tín hiệu quan trọng trong dữ liệu có
thể bị loại bỏ dưới dạng nhiễu}. Do đó, \textbf{nên cung cấp các chế độ
hiển thị dữ liệu thô trước} khi tạo ra các phiên bản mới, cho phép người
xem quyết định liệu sự dẫn xuất sang dữ liệu mới có phản ánh chính xác
dữ liệu ban đầu hay không.

\subsection{Phán đoán Tuyệt đối so với Phán đoán Tương
đối}\label{phuxe1n-ux111ouxe1n-tuyux1ec7t-ux111ux1ed1i-so-vux1edbi-phuxe1n-ux111ouxe1n-tux1b0ux1a1ng-ux111ux1ed1i}

Con người có khả năng tương đối \textbf{hạn chế trong việc} thực hiện
các \textbf{phán đoán tuyệt đối} về \textbf{kích thích thị giác}.

Hệ quả của điều này là những \textbf{phép trực quan phụ thuộc} quá nhiều
vào \textbf{việc người phải đo chính xác} các \textbf{thuộc tính đồ họa}
như vị trí, chiều dài và màu sắc sẽ \textbf{dẫn đến những vấn đề} trong
việc \textbf{diễn giải}.

\textbf{Cách khắc phục}:

\begin{itemize}
\item
  Thiết kế các hình ảnh trực quan \textbf{dựa vào phán đoán tương đối
  hơn là tuyệt đối}, hoặc sử dụng \textbf{thêm một vài số nhỏ riêng biệt
  cho mỗi đối tượng đồ họa} để truyền tải thông tin.
\item
  Các \textbf{hộp giới hạn}, \textbf{lưới} và \textbf{dấu vạch} đều là
  những công cụ tuyệt vời để \textbf{giúp chuyển việc phán đoán tuyệt
  đối} thành một nhiệm vụ \textbf{phụ thuộc nhiều hơn} vào \textbf{phán
  đoán tương đối}. Bằng cách \textbf{so sánh} chiều dài hoặc vị trí của
  \textbf{một thực thể đồ họa} cùng \textbf{với một cấu trúc đã được
  định lượng}, người xem có thể \textbf{nhanh chóng xác định giá trị gần
  đúng so với các mức} đã biết.
\item
  Sử dụng các \textbf{phần dư} (ví dụ, trừ các giá trị từ giá trị trung
  bình của chúng) cũng có thể \textbf{biến việc đo lường chính xác thành
  việc quyết định tương đối} liệu một giá trị nằm trên hoặc dưới một mức
  cụ thể nào đó.
\end{itemize}

\section{Tổng kết \& Các ứng dụng thực
tế}\label{tux1ed5ng-kux1ebft-cuxe1c-ux1ee9ng-dux1ee5ng-thux1ef1c-tux1ebf}

Như vậy, trong chương này, một số quy tắc thiết kế cho việc tạo ra hình
ảnh trực quan hiệu quả đã được trình bày, chúng bao gồm \textbf{quy
trình đảm bảo phù hợp các yếu tố}:

\begin{itemize}
\tightlist
\item
  \textbf{Ánh xạ} dữ liệu
\item
  \textbf{Chế độ} xem
\item
  \textbf{Mật độ} thông tin
\item
  Thông tin \textbf{chú thích}
\item
  \textbf{Màu sắc}
\item
  \textbf{Thẩm mỹ}
\end{itemize}

Ngoài ra, chúng tôi cung cấp thêm 1 số kỹ thuật để tránh các lỗi khác
sau khi hoàn thành quy trình trên:

\begin{itemize}
\tightlist
\item
  \textbf{Tránh đánh lừa người xem} bằng các tỷ lệ không cân bằng và các
  chiêu trò hình ảnh khác
\item
  Đảm bảo rằng \textbf{hình ảnh trực quan cần có ý nghĩa} ngữ nghĩa
\item
  Sử dụng lưới một cách hợp lý để \textbf{tránh che khuất dữ liệu}. Quá
  nhiều ``chart junk'' có thể \textbf{gây mất tập trung sự chú ý của
  người xem}
\item
  Cung cấp \textbf{quyền truy cập} vào \textbf{dữ liệu thô} cho người
  xem; thường thì việc làm sạch dữ liệu là chấp nhận được, nhưng người
  xem nên biết cách dữ liệu kết quả đã được tạo ra
\item
  Thiết kế hình ảnh trực quan \textbf{ưu tiên phán đoán tương đối} hơn
  là phán đoán tuyệt đối
\end{itemize}

Sau đây chúng ta hãy cùng thử ứng dụng những điều đã được học này vào dự
án cuối kỳ: Đánh giá lựa chọn danh mục đầu tư
